\documentclass[10pt]{beamer}
%\usetheme{hannover}
%\useinnertheme{circles}
\setbeamertemplate{footline}[page number]


\usepackage[english]{babel}
\usepackage{dsfont}
\usepackage[latin1]{inputenc}
\usepackage{amssymb}
\usepackage{times}
\usepackage[T1]{fontenc}


\newif\ifpdf
\ifx\pdfoutput\undefined
   \pdffalse
\else
   \pdfoutput=1
   \pdftrue
\fi
\ifpdf
   \usepackage{graphicx}
   \usepackage{epstopdf}
   \DeclareGraphicsRule{.eps}{pdf}{.pdf}{`epstopdf #1}
   \pdfcompresslevel=9
\else
   \usepackage{graphicx}
\fi





\title[Temp]{\textsc{The Macrodynamics of Sorting between Workers and Firms}}

\subtitle{\textsc{Lise and Robin} (2017)}

\author{\textit{Presented by:} Hong Ngoc Nguyen}

\date{March 27, 2019}


\begin{document}
 
\begin{frame}
  \titlepage
\end{frame}


\begin{frame}{Introduction}

This paper:

\begin{itemize}
\item develops an equilibrium model of on-the-job search with heterogeneous workers and firms and aggregate uncertainty.

\item proves that the model is very tractable.

\item illustrates the quantitative implications of the model by fitting to US aggregate labor market data from 1951-2012.

\item has rich implications for the cyclical dynamics of the distribution of vacancies, unemployed workers, and sorting between heterogeneous workers and firms.
\end{itemize}

\end{frame}


\begin{frame}{Advantages}

\begin{itemize}
\item w.r.t. existing equilibrium search models with heterogeneity: stochastic model is developed.

\item w.r.t. the directed search model: 
	\begin{itemize}
	\item two-sided heterogeneity is easily introduced; 
	\item search frictions generate mismatch at the equilibrium; 
	\item workers search on the job and employers counter outside offers; 
	\item decisions about wages and matching are separated.
	\end{itemize}

\item w.r.t. wage-posting models: how different workers match with different firms and the interaction between heterogeneity and aggregate shocks are described.


\end{itemize}

\end{frame}




\begin{frame}{Model}

\begin{itemize}
\item Heterogeneous workers $x$ and firms $y$; aggregate state $z_t$

\item $B_t(x)$: \textbf{\textit{value of unemployment}} to worker $x$ at $t$ 
\item $b(x, z_t)$: how much an unemployed worker $x$ earns at $t$ 
\item $W_{0, t}(x, y)$: value to worker $x$ hired from unemployment by firm $y$; $W_{0, t}(x, y) = B_t(x)$

\end{itemize}

\begin{figure}
    \includegraphics[width=\textwidth]{Eq1}
  \end{figure}

\begin{itemize}

 \begin{footnotesize}

\item $\lambda_t$: probability an unemployed searcher contacts a vacancy
\item $v_t(y)$: number of job opportunities chosen by firm $y$; $V_t = \int v_t(y) dy$

 \end{footnotesize}

\end{itemize}

\end{frame}



\begin{frame}{Model}

\begin{itemize}

\item $P_t(x, y)$: continuation \textbf{\textit{value of a match}} $(x, y)$; $p(x, y, z_t)$ at $t$

\end{itemize}

\begin{figure}
    \includegraphics[width=\textwidth]{Eq2}
  \end{figure}

\begin{itemize}


\item Incumbent and poaching firms engage in Bertrand competition which grants the worker the second highest bid.

	\begin{itemize} 
	\begin{footnotesize}
	\item If $P_{t+1}(x, y^\prime) >  P_{t+1}(x, y)$: $x$ moves to firm $y^\prime$ and receives $W_{1, {t+1}}(x, y^\prime, y) =  P_{t+1}(x, y)$; 
	\item If $P_{t+1}(x, y^\prime ) \leq  P_{t+1}(x, y)$: $x$ stays with firm $y$ and receives $W_{1, {t+1}}(x, y, y^\prime) =  P_{t+1}(x, y^\prime)$
	\end{footnotesize}
	\end{itemize}
	
\end{itemize}

\end{frame}



\begin{frame}{Model}

\begin{itemize}
\item \textbf{\textit{Match surplus}}: $S_t(x, y) = P_t(x, y) - B_t(x)$
\end{itemize}

\begin{figure}
    \includegraphics[width=\textwidth]{Eq3}
  \end{figure}

\begin{itemize}
\item The surplus depends on time only through $z_t$ and does not depend on the distributions of vacancies, unemployed workers, or worker-firm matches. 
\item Outside offers do not change the size of the match surplus. 
\item The surplus function fully characterizes the mobility decision of workers.
	
	\begin{itemize}
	\item For an unemployed worker: A match is formed if $S(x, y, z) > 0$
	\item For an employed worker: Poaching is successful if $S(x, y, z) > S(x, y^\prime, z)$ 
	\end{itemize}

\end{itemize}

\end{frame}






\begin{frame}{Fit}

\begin{itemize}
{\small \item Fit the model to moments of US time series data from 1951:I to 2012:IV}

\end{itemize}

\begin{figure}
    \includegraphics[width=0.7\textwidth]{Fig1}
  \end{figure}
  
\end{frame}



\begin{frame}{Job Creation and Job Separation}

\begin{figure}
    \includegraphics[width=0.8\textwidth]{Fig3}
  \end{figure}

\begin{itemize}
\begin{small}
\item Moving from a boom to a recession, the number of vacancies contracts everywhere, esp in low-type vacancies 
\item Overall, market production is substantially higher than home production, esp in high state => more posted vacancies
\item When home production is very close to market production, more in the low state, mismatched workers are at risk of endogenous separation
\end{small}

\end{itemize}

\end{frame}



\begin{frame}{Feasible Matches and Sorting}

\begin{figure}
    \includegraphics[width=0.6\textwidth]{Fig4}
  \end{figure}

\begin{itemize}
\begin{small}
\item The matching set is cone-shaped and sorting is strongly positive.
\item Lower-type workers have fewer employment opportunities, and workers with shorter employment tenure are more cyclically sensitive.
\item The firms' minimum worker type fluctuates substantially less than the workers' lowest firm type =>  matches between low-type firms and high-type workers are most at risk of endogenous separations.

\end{small}
\end{itemize}

\end{frame}




\begin{frame}{Equilibrium Distribution of Matches }

\begin{figure}
    \includegraphics[width=0.9\textwidth]{Fig5}
  \end{figure}

\begin{itemize}
\begin{small}
\item Substantial mass along the boundary relating to the firms' minimum worker type.
\item Fewer matches at the boundary in the good states: workers move more quickly to their preferred matches through on-the-job search.
\item On-the-job search results in the second ridge, the center of which corresponds to the optimal job for each worker.
\end{small}

\end{itemize}

\end{frame}




\begin{frame}{Business-Cycle Dynamics of Matches}

\begin{figure}
    \includegraphics[width=0.68\textwidth]{Fig6}
  \end{figure}

\end{frame}



\begin{frame}{Business-Cycle Dynamics of Matches}

\begin{itemize}
\begin{small}
\item Employment share: Expansion is largely the result of  low-/medium-type worker, low-type firm pairs.
\item Job separation rate: Low-type workers are the most susceptible in recession, esp those matched with high-type firms; High-type workers are completely shielded.
\item Share of hires: In recession, low-type firms hire less and medium-/high-type firms hire relatively more medium-/high-type unemployed workers.
\end{small}
 
\end{itemize}

\end{frame}
 
 


\begin{frame}{Extension}

\begin{itemize}
\item This model does not make any predictions about wages 
\item => Is this model able to also match wage data? 
\item => incorporate more direct empirical measures of worker and firm heterogeneity, such as measures based on education, occupation, wages, value added, and other conditional measures available in matched employer-employee data.
 
\end{itemize}

\end{frame}
 
 
\end{document}


